---
layout: post
title: LaTeX 数学公式测试
date: 2025-11-11
author: pepper
tags: [LaTeX, Math, Test]
comments: true
toc: true
pinned: false
---

# LaTeX 数学公式测试

本文用于测试本博客平台对 LaTeX 数学公式的支持情况。

## 行内公式

这是一个行内公式示例:$E = mc^2$,这是爱因斯坦的质能方程。

另一个行内公式:$\sum_{i=1}^{n} x_i = x_1 + x_2 + \cdots + x_n$

## 独立公式块

### 二次方程求根公式

$$x = \frac{-b \pm \sqrt{b^2 - 4ac}}{2a}$$

### 微积分

导数定义:
$$f'(x) = \lim_{h \to 0} \frac{f(x+h) - f(x)}{h}$$

积分:
$$\int_{a}^{b} f(x) dx = F(b) - F(a)$$

### 线性代数

矩阵乘法:
$$\begin{bmatrix}
a & b \\
c & d
\end{bmatrix}
\begin{bmatrix}
e & f \\
g & h
\end{bmatrix}
=
\begin{bmatrix}
ae + bg & af + bh \\
ce + dg & cf + dh
\end{bmatrix}$$

### 概率论

贝叶斯定理:
$$P(A|B) = \frac{P(B|A) \cdot P(A)}{P(B)}$$

正态分布:
$$f(x) = \frac{1}{\sigma\sqrt{2\pi}} e^{-\frac{(x-\mu)^2}{2\sigma^2}}$$

### 三角函数

欧拉公式:
$$e^{i\theta} = \cos\theta + i\sin\theta$$

### 求和与级数

无穷级数:
$$e^x = \sum_{n=0}^{\infty} \frac{x^n}{n!} = 1 + x + \frac{x^2}{2!} + \frac{x^3}{3!} + \cdots$$

### 分段函数

$$f(x) = \begin{cases}
x^2 & \text{if } x \geq 0 \\
-x^2 & \text{if } x < 0
\end{cases}$$

### 希腊字母

常用希腊字母:$\alpha, \beta, \gamma, \delta, \epsilon, \zeta, \eta, \theta, \iota, \kappa, \lambda, \mu, \nu, \xi, \omicron, \pi, \rho, \sigma, \tau, \upsilon, \phi, \chi, \psi, \omega$

大写希腊字母:$\Gamma, \Delta, \Theta, \Lambda, \Xi, \Pi, \Sigma, \Upsilon, \Phi, \Psi, \Omega$

## 复杂公式示例

### 麦克斯韦方程组

$$\begin{align}
\nabla \cdot \mathbf{E} &= \frac{\rho}{\epsilon_0} \\
\nabla \cdot \mathbf{B} &= 0 \\
\nabla \times \mathbf{E} &= -\frac{\partial \mathbf{B}}{\partial t} \\
\nabla \times \mathbf{B} &= \mu_0 \mathbf{J} + \mu_0 \epsilon_0 \frac{\partial \mathbf{E}}{\partial t}
\end{align}$$

### 薛定谔方程

$$i\hbar\frac{\partial}{\partial t}\Psi(\mathbf{r},t) = \hat{H}\Psi(\mathbf{r},t)$$

---

如果以上公式能够正常渲染,说明本博客平台完全支持 LaTeX 数学公式语法!





